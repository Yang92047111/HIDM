%%%%%%%%%%%%%%%%%%%%%%%%%%%%%%%%%%%%%%%%%
% Formal Book Title Page
% LaTeX Template
% Version 2.0 (23/7/17)
%
% This template was downloaded from:
% http://www.LaTeXTemplates.com
%
% Original author:
% Peter Wilson (herries.press@earthlink.net) with modifications by:
% Vel (vel@latextemplates.com)
%
% License:
% CC BY-NC-SA 3.0 (http://creativecommons.org/licenses/by-nc-sa/3.0/)
% 
% This template can be used in one of two ways:
%
% 1) Content can be added at the end of this file just before the \end{document}
% to use this title page as the starting point for your document.
%
% 2) Alternatively, if you already have a document which you wish to add this
% title page to, copy everything between the \begin{document} and
% \end{document} and paste it where you would like the title page in your
% document. You will then need to insert the packages and document 
% configurations into your document carefully making sure you are not loading
% the same package twice and that there are no clashes.
%
%%%%%%%%%%%%%%%%%%%%%%%%%%%%%%%%%%%%%%%%%

%----------------------------------------------------------------------------------------
%	PACKAGES AND OTHER DOCUMENT CONFIGURATIONS
%----------------------------------------------------------------------------------------

\documentclass[a4paper, 11pt, oneside]{book} % A4 paper size, default 11pt font size and oneside for equal margins

\newcommand{\plogo}{\fbox{$\mathcal{PL}$}} % Generic dummy publisher logo

\usepackage[utf8]{inputenc} % Required for inputting international characters
\usepackage[T1]{fontenc} % Output font encoding for international characters
\usepackage{fouriernc} % Use the New Century Schoolbook font
\usepackage{amsmath}

%----------------------------------------------------------------------------------------
%	TITLE PAGE
%----------------------------------------------------------------------------------------

\begin{document} 

\begin{titlepage} % Suppresses headers and footers on the title page

	\centering % Centre everything on the title page
	
	\scshape % Use small caps for all text on the title page
	
	\vspace*{\baselineskip} % White space at the top of the page
	
	%------------------------------------------------
	%	Title
	%------------------------------------------------
	
	\rule{\textwidth}{1.6pt}\vspace*{-\baselineskip}\vspace*{2pt} % Thick horizontal rule
	\rule{\textwidth}{0.4pt} % Thin horizontal rule
	
	\vspace{0.75\baselineskip} % Whitespace above the title
	
	{\LARGE Human-Centered Information \\ and \\ Data Mining} % Title
	
	\vspace{0.75\baselineskip} % Whitespace below the title
	
	\rule{\textwidth}{0.4pt}\vspace*{-\baselineskip}\vspace{3.2pt} % Thin horizontal rule
	\rule{\textwidth}{1.6pt} % Thick horizontal rule
	
	\vspace{2\baselineskip} % Whitespace after the title block
	
	%------------------------------------------------
	%	Subtitle
	%------------------------------------------------
	
	Homework 1 % Subtitle or further description
	
	\vspace*{3\baselineskip} % Whitespace under the subtitle
	
	%------------------------------------------------
	%	Editor(s)
	%------------------------------------------------
	
	Student: 
	
	\vspace{0.5\baselineskip} % Whitespace before the editors

	{\scshape 108368017 \hspace{10mm}}
	{\scshape\Large Zi-Yang Lin } % Editor list

	\vspace{0.5\baselineskip} % Whitespace before the editors

	Advisor:

	\vspace{0.5\baselineskip} % Whitespace before the editors
	
	{\scshape\Large Jenq-Haur Wang} % Editor list
	
	\vspace{0.5\baselineskip} % Whitespace below the editor list
	
	\textit{National Taipei University of Technology} % Editor affiliation
	
	\vfill % Whitespace between editor names and publisher logo
	
	%------------------------------------------------
	%	Publisher
	%------------------------------------------------
	
	% \plogo % Publisher logo
	
	\vspace{0.3\baselineskip} % Whitespace under the publisher logo
	
	2019 % Publication year
	
	% {\large publisher} % Publisher

\end{titlepage}

%----------------------------------------------------------------------------------------

	\clearpage

	2.4: Suppose that a hospital tested the age and body fat data for 18 \\ randomly selected adults with the following results:

	\vspace{0.5\baselineskip}

	\begin{center}
	\begin{tabular}{ |c|c|c|c|c|c|c|c|c|c|c| } 
	\hline

	Age & 23 & 23 & 27 & 27 & 39 & 41 & 47 & 49 & 50 \\ 
	\%fat & 9.5 & 26.5 & 7.8 & 17.8 & 31.4 & 25.9 & 27.4 & 27.2 & 31.2 \\ 

	\hline
	\end{tabular}
	\end{center}

	\vspace{0.5\baselineskip}

	\begin{center}
	\begin{tabular}{ |c|c|c|c|c|c|c|c|c|c|c| } 
	\hline
	
	Age & 52 & 54 & 54 & 56 & 57 & 58 & 58 & 60 & 61 \\ 
	\%fat & 34.6 & 42.5 & 28.8 & 33.4 & 30.2 & 34.1 & 32.9 & 41.2 & 35.7 \\ 
	
	\hline
	\end{tabular}
	\end{center}

	\begin{description}

		% \vspace{0.5\baselineskip}

		\item (a) Calculate the mean, and median of age and \%fat.

		\vspace{0.5\baselineskip}

		Ans:

		\begin{description}

			% \vspace{0.5\baselineskip}

			\item Age mean: 46.44444444444444,
			\item Age median: 51.0

			\vspace{0.5\baselineskip}

			\item fat mean: 28.783333333333328,
			\item fat median: 30.7
		
		\end{description}
	
	\end{description}

	\vspace{0.5\baselineskip}

	\clearpage

	2.8: It is important to define or select similarity measures in data analysis.
	However, there is no commonly accepted subjective similarity measure.
	Results van vary depending on the similarity measures used.
	Nonetheless, seemingly different similarity measures may be equivalent after some transformation.
	Suppose we have the following 2-D data set:

	\vspace{0.5\baselineskip}

	\begin{center}
	\begin{tabular}{||c c c||} 
	\hline
	  & A1 & A2 \\ [0.5ex] 
	\hline\hline
	x1 & 1.5 & 1.7 \\ 
	\hline
	x2 & 2 & 1.9 \\
	\hline
	x3 & 1.6 & 1.8 \\
	\hline
	x4 & 1.2 & 1.5 \\
	\hline
	x5 & 1.5 & 1.0 \\ [1ex] 
	\hline
	\end{tabular}
	\end{center}

	\begin{description}
		\item (a) Consider the data as 2-D data points. Given a new data point, x=(1.4,1.6) as a query,
		rank the database points based on similarity with the query using Euclidean distance,
		Manhattan distance, supremum distance, and cosine similarity.

		\vspace{0.5\baselineskip}

		Ans:

		\begin{description}

			\item Euclidean distance: (x2, 0.67) > (x5, 0.608) > (x3, 0.28) > (x4, 0.22) > (x1, 0.14)
			\vspace{0.5\baselineskip}
			\item Manhattan distance: (x2, 0.89) > (x5, 0.7) > (x3, 0.4) > (x4, 0.3) > (x1, 0.19)
			\vspace{0.5\baselineskip}
			\item Supremum distance: (x2, 0.6) = (x5, 0.6) > (x3, 0.2) > (x4, 0.19) > (x1, 0.1)
			\vspace{0.5\baselineskip}
			\item Cosine similarity: (x1, 0.99999) > (x3, 0.99996) > (x4, 0.999) > (x2, 0.995) > (x5, 0.96)
			
		\end{description}

	\end{description}

	\clearpage
	
	3.8: Using the data for age and body fat given in Exercise 2.4, answer the following:

	\vspace{0.5\baselineskip}

	\begin{description}

		\item (a) Normalize the two attributes based on z-score normalization.
		\item (b) Calculate the correlation coefficient (Pearson’s product moment coefficient).
		Are these attributes positively or negatively correlated?
		Compute their covariance.

		\vspace{0.5\baselineskip}

		Ans:

		\begin{description}
			\item (a)
		\end{description}

		\begin{center}
		\begin{tabular}{ |c|c|c|c|c|c|c|c|c|c|c| }
			\hline
			-1.825 & -1.825 & -1.513 & -1.513 & -0.579 & -0.423 & 0.043 & 0.198 & 0.276 \\
			\hline
			0.432 & 0.588 & 0.588 & 0.743 & 0.821 & 0.899 & 0.899 & 1.055 & 1.133 \\
			\hline
		\end{tabular}
		\end{center}
			
		\begin{description}
		
			\item (b)
			Pearson’s correlation coefficient: 0.817, so age and \%fat are \\ positively correlated.
		
		\end{description}
		
		$$
		covariance = 
		\begin{bmatrix}
			174.732 & 100.0196 \\
			100.0196 & 85.643
		\end{bmatrix}
		$$
		
	\end{description}

	\clearpage

	3.9: Suppose a group of 12 sales price records has been sorted as follows:

	\vspace{0.5\baselineskip}

	\begin{center}
	\begin{tabular}{ |c|c|c|c|c|c| } 
	\hline

	5 & 10 & 11 & 13 & 15 & 35 \\ 
	50 & 55 & 72 & 92 & 204 & 215 \\

	\hline
	\end{tabular}
	\end{center}

	Partition them into three bins by each of the following methods:

	\begin{description}
		\item (a) equal-frequency (equal-depth) partitioning
		\item (b) equal-width partitioning
	\end{description}

	\vspace{0.5\baselineskip}

	Ans:

	\begin{description}
		\item (a)
		\item equal-frequency partitioning: 
	\end{description}

	$$
	\begin{bmatrix}
		5 & 10 & 11 & 13
	\end{bmatrix}
	$$
	$$
	\begin{bmatrix}
		15 & 35 & 50 & 55
	\end{bmatrix}
	$$
	$$
	\begin{bmatrix}
		72 & 92 & 204 & 215
	\end{bmatrix}
	$$

	\begin{description}
		\item (b)
		\item equal-width partitioning: 
	\end{description}

	$$
	\begin{bmatrix}
		10 & 11 & 13 & 15 & 35 & 50 & 55 & 72
	\end{bmatrix}
	$$
	$$
	\begin{bmatrix}
		92
	\end{bmatrix}
	$$
	$$
	\begin{bmatrix}
		204
	\end{bmatrix}
	$$

\end{document}
